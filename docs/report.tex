%%%%%%%%%%%%%%%%%%%%%%%%%%%%%%%%%%%%%%%%%%%%%%%%%%%%%%%%%%%%%%%%%%%%%%%%%%%%%%%%
%2345678901234567890123456789012345678901234567890123456789012345678901234567890
%        1         2         3         4         5         6         7         8

\documentclass[letter, 12pt, conference]{ieeeconf}

\overrideIEEEmargins
% See the \addtolength command later in the file to balance the column lengths
% on the last page of the document

% Reduce space between figures and text
\setlength{\textfloatsep}{5pt}

\usepackage[utf8]{inputenc}
\usepackage[T1]{fontenc}
\usepackage{float}
\usepackage{graphics}
\usepackage{graphicx}
\usepackage{nth}
\usepackage{pgfplots}
\usepackage{titlesec}
\usepackage[backend=biber,style=ieee]{biblatex}
\usepackage{multirow}

\newcommand{\todo}{\colorbox{red}{TODO}}
\newcommand{\todocite}{\colorbox{red}{TODO\_CITE}}

\addbibresource{references.bib}
\renewcommand*{\bibfont}{\small}

\title{\LARGE \bf Practical Analysis of Hybrid Sorting Algorithms}
\author{Joshua Arulsamy}

\author{\parbox{3 in}{
	\centering
	Joshua Arulsamy\\
    University of Wyoming\\
    {\tt\small jarulsam@uwyo.edu}}}

\begin{document}

\maketitle
\thispagestyle{plain}
\pagestyle{plain}
\nocite{*}

%%%%%%%%%%%%%%%%%%%%%%%%%%%%%%%%%%%%%%%%%%%%%%%%%%%%%%%%%%%%%%%%%%%%%%%%%%%%%%%%
\begin{abstract}

  \todo

\end{abstract}

%%%%%%%%%%%%%%%%%%%%%%%%%%%%%%%%%%%%%%%%%%%%%%%%%%%%%%%%%%%%%%%%%%%%%%%%%%%%%%%%
\section{INTRODUCTION}

Most applications require data to be arranged in a certain order. The procedure
of reordering data according to some comparator is called sorting. Sorting can
be found in nearly every computer system today. From databases to payment
processing systems, data is constantly being sorted. The near omnipresent need
for performant sorting implementations has resulted in all major programming
languages including a generic sorting function within their standard libraries.
These standard implementations can use a variety of algorithms internally, such
as Quicksort, Mergesort, and many others (\todocite), however, rarely does a
single algorithm perform exceptionally well. Often, these algorithms are paired
with one or more other sorting algorithms to offer sizeable performance
benefits. These hybrid algorithms include some tuneable parameters, such as when
to switch from one sorting algorithm to another. While these hybridized
algorithms have varying space and time complexities, their actual performance in
real-world scenarios is rarely formally examined, and the values for these
various parameters are often chosen arbitrarily by the developer. This paper
analyzes a wide variety of sorting algorithm implementations across varying
system architectures to determine optimal parameter configurations for hybrid
sorting algorithms. This paper also proposes a new, performant standard sorting
implementation for GNU's C standard library.

\section{BACKGROUND}

Substantial effort has been spent on analysis of sorting algorithms. Typically,
two major characteristics are considered when studying sorting algorithms from a
purely theoretical standpoint, space and time complexity. Existing analysis of
popular sorting algorithms such as Quicksort and Mergesort place substantial
weight on their excellent asymptotic complexity in the average case(\todocite).
However, rarely does asymptotic complexity alone determine the practical
performance of an algorithm.

\begin{table}[]
\begin{tabular}{|c|ccc|c|}
\hline
\textbf{Algorithm} & \multicolumn{3}{c|}{\textbf{Time Complexity}}                                               & \textbf{Space Complexity} \\ \hline
                   & \multicolumn{1}{c|}{\textbf{Best}} & \multicolumn{1}{c|}{\textbf{Average}} & \textbf{Worst} & \textbf{Worst}            \\ \hline
Quicksort          & \multicolumn{1}{c}{$\Omega(n\log{(n)})$}          & \multicolumn{1}{c}{$\Theta(n\log{(n)})$}             & todo           & todo                      \\ \hline
Mergesort          & \multicolumn{1}{c}{$\Omega(n\log{(n)})$}          & \multicolumn{1}{c}{$\Theta(n\log{(n)})$}             & todo           & todo                      \\ \hline
Heapsort           & \multicolumn{1}{c}{$\Omega(n\log{(n)})$}          & \multicolumn{1}{c}{$\Theta(n\log{(n)})$}             & todo           & todo                      \\ \hline
Bubble Sort        & \multicolumn{1}{c}{$\Omega(n)$}                   & \multicolumn{1}{c}{$\Theta(n^{2})$}             & todo           & todo                      \\ \hline
Insertion Sort     & \multicolumn{1}{c}{$\Omega(n)$}                   & \multicolumn{1}{c}{$\Theta(n^{2})$}             & todo           & todo                      \\ \hline
Tree Sort          & \multicolumn{1}{c}{$\Omega(n\log{(n)})$}          & \multicolumn{1}{c}{$\Theta((n\log{(n)}))^{2}$}             & todo           & todo                      \\ \hline
Shell Sort         & \multicolumn{1}{c}{$\Omega(n\log{(n)})$}          & \multicolumn{1}{c}{todo}             & todo           & todo                      \\ \hline
\end{tabular}
\end{table}

\section{EXPERIMENTAL SETUP}

\todo

\section{RESULTS}

\todo

\section{CONCLUSION}

\todo

%%%%%%%%%%%%%%%%%%%%%%%%%%%%%%%%%%%%%%%%%%%%%%%%%%%%%%%%%%%%%%%%%%%%%%%%%%%%%%%%

% This command serves to balance the column lengths on the last page of the
% document manually. It shortens the textheight of the last page by a suitable
% amount. This command does not take effect until the next page so it should
% come on the page before the last. Make sure that you do not shorten the
% textheight too much.
% \addtolength{\textheight}{-12cm}

% \titleformat*{\section}{\fontsize{12pt}{14pt}\MakeUppercase\selectfont}
\printbibliography
\end{document}
