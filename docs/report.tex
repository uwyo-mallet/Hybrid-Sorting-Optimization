%%%%%%%%%%%%%%%%%%%%%%%%%%%%%%%%%%%%%%%%%%%%%%%%%%%%%%%%%%%%%%%%%%%%%%%%%%%%%%%%
%2345678901234567890123456789012345678901234567890123456789012345678901234567890
%        1         2         3         4         5         6         7         8

\documentclass[12pt, conference]{ieeeconf}

\overrideIEEEmargins
% See the \addtolength command later in the file to balance the column lengths
% on the last page of the document

% Reduce space between figures and text
\setlength{\textfloatsep}{5pt}

\usepackage[utf8]{inputenc}
\usepackage[T1]{fontenc}
\usepackage{float}
\usepackage{graphics}
\usepackage{graphicx}
\usepackage{nth}
\usepackage{pgfplots}
\usepackage{titlesec}
\usepackage[backend=biber,style=ieee]{biblatex}
\usepackage{multirow}
\usepackage{algorithm}
\usepackage{algpseudocode}

\newcommand{\todo}{\colorbox{red}{TODO}}
\newcommand{\todocite}{\colorbox{red}{CITE}}

\addbibresource{references.bib}
\renewcommand*{\bibfont}{\small}

% Defaults
% \setlength{\textfloatsep}{10pt plus 1.0pt minus 2.0pt}
% \setlength{\floatsep}{12pt plus 2.0pt minus 2.0pt}
% \setlength{\intextsep}{12pt plus 2.0pt minus 2.0pt}

% \setlength{\textfloatsep}{0pt plus 1.0pt minus 2.0pt}
% \setlength{\floatsep}{12pt plus 2.0pt minus 2.0pt}
\setlength{\intextsep}{5pt plus 2.0pt minus 2.0pt}

\title{\LARGE \bf Practical Analysis of Hybrid Sorting Algorithms}
\author{Joshua Arulsamy}

\author{\parbox{3 in}{
\centering
Joshua Arulsamy\\
University of Wyoming\\
{\tt\small jarulsam@uwyo.edu}}}

\begin{document}

\maketitle
\thispagestyle{plain}
\pagestyle{plain}
\nocite{*}

%%%%%%%%%%%%%%%%%%%%%%%%%%%%%%%%%%%%%%%%%%%%%%%%%%%%%%%%%%%%%%%%%%%%%%%%%%%%%%%%
\begin{abstract}

	\todo

\end{abstract}

%%%%%%%%%%%%%%%%%%%%%%%%%%%%%%%%%%%%%%%%%%%%%%%%%%%%%%%%%%%%%%%%%%%%%%%%%%%%%%%%
\section{INTRODUCTION}

Most applications require data to be arranged in a certain order. The procedure
of reordering data according to some comparator is called sorting. Sorting can
be found in nearly every computer system today. From databases to payment
processing systems, data is constantly being sorted. The near omnipresent need
for performant sorting implementations has resulted in most major programming
languages including a type-generic sorting function within their standard
libraries. These standard implementations can use a variety of algorithms
internally, such as Quicksort, Mergesort, and many others, however, rarely does
a single algorithm perform exceptionally well. Often, these algorithms are
paired with one or more other sorting algorithms to offer sizeable performance
benefits. These hybrid algorithms include some tuneable parameters, such as when
to switch from one sorting algorithm to another. While these hybridized
algorithms have varying space and time complexities, their actual performance in
real-world scenarios is rarely formally examined, and the values for these
parameters are often chosen arbitrarily by the developers. This paper analyzes a
wide variety of sorting algorithm implementations across varying system
architectures to determine optimal parameter configurations for hybrid sorting
algorithms. This paper also proposes a new performant standard sorting
implementation for GNU's C standard library leveraging Mergesort, insertion
sort, and sorting networks.

\section{BACKGROUND}

Substantial effort has been spent on the analysis of sorting algorithms.
Typically, two major characteristics are considered when studying sorting
algorithms from a purely theoretical standpoint; space and time complexity.
Existing analysis of popular sorting algorithms, such as Quicksort and
Mergesort, place substantial emphasis on their excellent asymptotic complexity
in the average case\parencite{glibc}. However, asymptotic complexity alone does
not describe the practical performance of these algorithms. While Mergesort may
have a better asymptotic time complexity in the worst case than insertion sort
(Table \ref{fig:time_complexity_table}), for small inputs, insertion sort
typically outperforms Mergesort by avoiding the overhead of recursive calls.

\begin{table}[ht]
	\centering
	\resizebox{\columnwidth}{!}{
		\begin{tabular}{|c|lll|}
			\hline
			\textbf{Algorithm} & \multicolumn{3}{c|}{\textbf{Time Complexity}}                                                                                 \\ \hline
			                   & \multicolumn{1}{l|}{\textbf{Best}}            & \multicolumn{1}{l|}{\textbf{Average}}         & \textbf{Worst}                \\ \hline
			Quicksort          & \multicolumn{1}{l|}{$\Omega(n\log{(n)})$}     & \multicolumn{1}{l|}{$\Theta(n\log{(n)})$}     & $\mathcal{O}(n^{2})$          \\ \hline
			Mergesort          & \multicolumn{1}{l|}{$\Omega(n\log{(n)})$}     & \multicolumn{1}{l|}{$\Theta(n\log{(n)})$}     & $\mathcal{O}(n\log{(n)})$     \\ \hline
			Heapsort           & \multicolumn{1}{l|}{$\Omega(n\log{(n)})$}     & \multicolumn{1}{l|}{$\Theta(n\log{(n)})$}     & $\mathcal{O}(n\log{(n)})$     \\ \hline
			Bubble Sort        & \multicolumn{1}{l|}{$\Omega(n)$}              & \multicolumn{1}{l|}{$\Theta(n^{2})$}          & $\mathcal{O}(n\log{(n)})$     \\ \hline
			Insertion Sort     & \multicolumn{1}{l|}{$\Omega(n)$}              & \multicolumn{1}{l|}{$\Theta(n^{2})$}          & $\mathcal{O}(n^{2})$          \\ \hline
			Tree Sort          & \multicolumn{1}{l|}{$\Omega(n\log{(n)})$}     & \multicolumn{1}{l|}{$\Theta(n\log(n))$}       & $\mathcal{O}(n^{2})$          \\ \hline
			Shell Sort         & \multicolumn{1}{l|}{$\Omega(n\log{(n)})$}     & \multicolumn{1}{l|}{$\Theta(n(\log(n))^{2})$} & $\mathcal{O}(n(\log(n))^{2})$ \\ \hline
		\end{tabular}}
	\caption{Sorting algorithm asymptotic time complexities}
	\label{fig:time_complexity_table}
\end{table}

Most of these algorithms have not been sufficiently benchmarked in real world
scenarios to determine an optimal implementation for common inputs. Evaluating
specific scenarios where an algorithm performs well may lead to better sorting
function implementations by combining several algorithms or excluding algorithms
in specific situations. Since sorting is used so ubiquitously, even incremental
improvements to standard library sorting implementations can broadly influence
the performance of many different applications. This marks an area in obvious
need of further investigation, and such an analysis does not presently exist.


\subsection{General Purpose Algorithms}

Of the well known sorting algorithms, several are used for general sorting
tasks, such as Quicksort, Mergesort, Heapsort, and Introsort. Quicksort has been
a staple algorithm for decades. It is well known to be easy to implement and
works well for a variety of input data. Quicksort requires a constant amount of
extra memory independent of the size of the input, making it an obvious choice
in resource constrained environments. However, in the advent of modern computing
where using auxiliary memory to improve performance is much more common,
Quicksort tends to fall behind other algorithms. Quicksort's performance is also
primarily dictated by the selection of a suitable pivot value. Poor pivot
selections can lead to worse performance. Simple implementations usually choose
the left-most value of the input as the pivot, however this offers poor
performance in many circumstances, for example, if the input is reverse sorted.
So most implementations use a slightly more complex and costly pivot selection
process such as \textit{median-of-three}\todocite.

The increased memory availability of modern systems and the difficulty of good
pivot selection within Quicksort has motivated many standard libraries -- such
as GNU's libc -- to default to using Mergesort, falling back to Quicksort only
when memory requirements are too high\parencite{glibc}. Such implementations are
typically more performant in most cases, since Mergesort can leverage extra
memory for better performance, and maintain usability even in memory constrained
environments.

\subsection{Specialized Algorithms}

A few algorithms used in specialized circumstances are also evaluated, most
notably, various sorting network implementations, insertion sort, and shell
sort. These algorithms are often regarded as the best sorting algorithms for
extremely small inputs. Since their asymptotic complexity is substantially worse
than a general purpose sorting algorithm their runtime typically grows
exponentially as the input size increases (Table
\ref{fig:time_complexity_table}). However, for small inputs, these algorithms
are highly performant since they avoid the overhead of more complicated
algorithms.

Sorting networks, also commonly referred to as comparator networks, are designed
to sort fixed numbers of items. Unlike other sorting algorithms, networks cannot
handle arbitrarily large inputs. However, even with this constraint, sorting
networks are still commonly used. They are highly performant in instances where
the input is extremely small. Since the order at which elements are compared is
always constant, carefully designing an optimal sorting network which minimizes
the number of comparisons will likely benefit from modern CPU branch
predictions. For small input sizes ($n < 16$), minimal comparison optimal
sorting networks have been thoroughly investigated, and their performance is
proven\parencite{knuth_networks}\parencite{engineering_fast_sorters}. For
slightly larger inputs, insertion sort is a prime candidate. Unlike Mergesort
and Quicksort, insertion sort is simple. The minimal overhead typically allows
insertion sort to outperform more complicated algorithms for small inputs. These
algorithms are prime candidates for hybridization with general purpose
algorithms.


\section{Proposed Algorithm}

General purpose sorting algorithms are typically difficult to optimize
uniformly. Since these implementations very often live in standard libraries,
use-case specific optimizations are not feasible. During runtime, quick,
low-overhead analysis about the input can reveal opportunities for
input-specific optimizations. More esoteric sorting methods, such as TimSort,
take advantage of such analysis during sorting to improve performance in most
real-world cases. However, for small input sizes, the overhead from such an
analysis often contributes more to the overall runtime, than even utilizing an
algorithm ill-suited for that specific input\parencite{the_basic_algorithms}.
This motivates placing significant importance on the size of the input itself.

Comparing the input size against predefined threshold values is cheap and can
help pick an algorithm. Since such comparisons are so cheap, they can be
repeated during recursive calls of other algorithms. This algorithm checks the
size of each subarray during Mergesort before switching to insertion sort for
small arrays. In instances where a subarray contains less than 5 elements, a
network sorting algorithm is used instead of insertion sort. Utilizing secondary
algorithms limits the depth of recursive calls substantially. Practically, this
reduces a significant amount of the overhead of Mergesort, even though there are
additional size comparisons during each recursive
call\parencite{the_basic_algorithms}.

\subsection{Implementation}

The actual implementation of the proposed algorithms differ slightly from the
psuedocode descriptions. Since this investigation is focused on improving
standard library implementations, specifically GNU libc's \verb|qsort()|, all of
these algorithms are implemented in C in a type-generic fashion in accordance
with ISO C standard\parencite{iso_c}. This presents another opportunity for
optimization. Typically, type-generic functions in C are implemented using
\verb|void| pointers -- essentially subverting the type system itself. Such
implementations must take care in maintaining alignment when moving any elements
within a contiguous block of un-typed memory. The catch-all method for safely
manipulating un-typed memory is resorting to \verb|memcpy| and its family of
system calls. However, system calls are extremely expensive and repeated usage
of \verb|memcpy| has a significant performance penalty. This implementation
avoids repeated system calls. Upon invocation, several preliminary heuristics
are immediately examined; size of each element within the input, total input
length, and alignment.

If the size of each individual element is large, indirect sorting is utilized. A
new array is populated with pointers to the original inputs and the pointers are
ordered according to the comparator function. Once the pointers are correctly
ordered, the elements in the original array are copied to the correct location
based on the position of their corresponding indirect sorting pointer. This way,
swaps stay cheap, since the amount of total memory being moved remains less than
8 bytes per element per swap.

The size of each element and total length of input are used to calculate the
total required auxiliary temporary space for Mergesort. If the required memory
is sufficiently small, it is stack allocated, otherwise it is heap allocated. If
at any time a request for memory fails, Mergesort is abandoned and Quicksort is
utilized. Using some clever pointer arithmetic, the alignment of the input data
is matched against several pre-defined alignments. Depending on the alignment,
the input data can be treated as a known primitive, such as an 32-bit unsigned
integer, making swaps of type-generic inputs essentially cost the same as
swapping primitive integers. In practice, this offers a very sizeable
performance gain.

\begin{algorithm}[ht]
	\caption{Network Sort}
	\label{alg:network_sort}
	\begin{algorithmic}
		\Procedure{SORT2}{$x$, $y$}
		\State $minPtr \gets \Call{min}{x, y}$
		\If {$x = minPtr$}
		\State $maxPtr \gets y$
		\Else
		\State $maxPtr \gets x$
		\EndIf
		\State $t \gets minPtr$
		\State $y \gets maxPtr$
		\State $x \gets t$
		\EndProcedure

		\Procedure{SORT3}{$a$}
		\State \Call{SORT2}{$a[0]$, $a[2]$}
		\State \Call{SORT2}{$a[0]$, $a[1]$}
		\State \Call{SORT2}{$a[1]$, $a[2]$}
		\EndProcedure

		\Procedure{SORT4}{$a$}
		\State \Call{SORT2}{$a[0]$, $a[2]$}
		\State \Call{SORT2}{$a[1]$, $a[3]$}
		\State \Call{SORT2}{$a[0]$, $a[1]$}
		\State \Call{SORT2}{$a[2]$, $a[3]$}
		\State \Call{SORT2}{$a[1]$, $a[2]$}
		\EndProcedure
	\end{algorithmic}
\end{algorithm}

\begin{algorithm}[ht]
	\caption{Insertion Sort}
	\label{alg:insertion_sort}
	\begin{algorithmic}
		\Procedure{INS\_SORT}{a, n}
		\For {$i \gets 1$ to $n$}
		\State $c \gets 1$
		\State $j \gets i - 1$
		\If {$a[j] > a[i]$}
		\State $tmp \gets a[i]$
		\Repeat
		\State $a[j + 1] \gets a[j]$
		\If {$j = 0$}
		\State $c \gets 0$
		\State \textbf{break}
		\EndIf
		\State $j \gets j - 1$
		\Until $a[j] \le tmp$
		\State $a[j + c] \gets tmp$
		\EndIf
		\EndFor
		\EndProcedure
	\end{algorithmic}
\end{algorithm}

\begin{algorithm}[ht]
	\caption{Hybridized Merge Sort}
	\label{alg:merge_sort}
	\begin{algorithmic}
		\Procedure{msort}{$a$, $n$, $threshold$}
		\If {$n \le 1$}
		\State \Return
		\EndIf

		\If {$n < threshold$}
		\If {$n = 2$}
		\State \Call{sort2}{$a$, $n$}
		\State \Return
		\ElsIf {$n = 3$}
		\State \Call{sort3}{$a$, $n$}
		\State \Return
		\ElsIf {$n = 4$}
		\State \Call{sort4}{$a$, $n$}
		\State \Return
		\EndIf

		\State \Call{ins\_sort}{$a$, $n$}
		\State \Return
		\EndIf

		\State $n_{1} \gets \frac{n}{2}$
		\State $n_{2} \gets n - n_{1}$
		\State $b_{1} \gets a$
		\State $b_{2} \gets \&b[n_{1}]$

		% Recursively sort two halves.
		\State \Call{msort}{$b_{1}$, $n_{1}$, $threshold$}
		\State \Call{msort}{$b_{2}$, $n_{2}$, $threshold$}

		% Merge the two halves.
		\While {$n_{1} > 0$ and $n_{2} > 0$}
		\If {$b_{1}[0] \le b_{2}[0]$}
		\State $tmp \gets b_{1}$
		\State $b_{1} \gets \&b_{1}[1]$
		\State $n_{1} \gets n_{1} - 1$
		\Else
		\State $tmp \gets b_{2}$
		\State $b_{2} \gets \&b_{2}[1]$
		\State $n_{2} \gets n_{2} - 1$
		\EndIf
		\EndWhile

		\If {$n_{1} > 0$}
		\State copy($tmp$, $b_{1}$, $n_{1}$)
		\EndIf
		\EndProcedure
		\State copy($a$, $tmp$, $n - n_{2}$)
	\end{algorithmic}
\end{algorithm}

% \section{EXPERIMENTAL SETUP}

A small subset of algorithms were chosen for evaluation. Primarily, algorithms
utilized within standard library implementations are of particular interest,
since they are typically highly performant in the general case. Other
complementary algorithms which perform exceptionally well in specific scenarios
were also evaluated. Since this analysis is focused on analyzing and improving
existing implementations, general purpose sorting function were taken from
various standard libraries, such as GNU's libc\parencite{glibc} and musl
libc\parencite{musl_libc}. It is a priority to maintain the applicability of any
optimizations discovered as a result of this evaluation to existing standard
libraries, so any non-comparison based sorting algorithm, such as Radix sort or
Bucket sort, were excluded. Other hybridization combinations were also included,
such as combining Mergesort with just insertion sort, to serve as a comparison.

\section{RESULTS}

\todo

\section{CONCLUSION}

\todo

%%%%%%%%%%%%%%%%%%%%%%%%%%%%%%%%%%%%%%%%%%%%%%%%%%%%%%%%%%%%%%%%%%%%%%%%%%%%%%%%

% This command serves to balance the column lengths on the last page of the
% document manually. It shortens the textheight of the last page by a suitable
% amount. This command does not take effect until the next page so it should
% come on the page before the last. Make sure that you do not shorten the
% textheight too much.
% \addtolength{\textheight}{-12cm}

% \titleformat*{\section}{\fontsize{12pt}{14pt}\MakeUppercase\selectfont}
\printbibliography
\end{document}
