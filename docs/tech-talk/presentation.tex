\documentclass[13pt]{beamer}

\usetheme{CambridgeUS}
\usepackage{nth}
\usepackage{pgfplots}
\usepackage{csvsimple}
\usepackage{graphicx}

\usepackage[backend=biber,style=ieee]{biblatex}
% \addbibresource{references.bib}

\title{Practical Analysis of Hybrid Sorting Algorithms}
\subtitle{Engineering a Faster Standard Sorting Implementation}
\author{Joshua Arulsamy}
\date{May 18, 2023}

\makeatletter
\defbeamertemplate*{footline}{Dan P theme}
{
  \leavevmode%
  \hbox{%
  \begin{beamercolorbox}[wd=.3\paperwidth,ht=2.25ex,dp=1ex,center]{author in head/foot}%
    \usebeamerfont{author in head/foot}\insertshortauthor\expandafter\beamer@ifempty\expandafter{\beamer@shortinstitute}{}{~~(\insertshortinstitute)}
  \end{beamercolorbox}%
  \begin{beamercolorbox}[wd=.4\paperwidth,ht=2.25ex,dp=1ex,center]{title in head/foot}%
    \usebeamerfont{title in head/foot}\insertshorttitle
  \end{beamercolorbox}%
  \begin{beamercolorbox}[wd=.3\paperwidth,ht=2.25ex,dp=1ex,right]{date in head/foot}%
    \usebeamerfont{date in head/foot}\insertshortdate{}\hspace*{2em}
\insertframenumber{} / \inserttotalframenumber\hspace*{2ex}
  \end{beamercolorbox}}%
  \vskip0pt%
}
\makeatother

\begin{document}

% Remove logo
\logo{}
\nocite{*}

\begin{frame}
	\titlepage
\end{frame}

\section{}
\subsection{}
\begin{frame}{Outline}
	\tableofcontents
\end{frame}

\section{Introduction}
\begin{frame}{Introduction}
	TODO
\end{frame}

\section{Motivation}
\subsection{What's the Deal with Sorting?}
\begin{frame}{What's the Deal with Sorting?}
	TODO
\end{frame}

\subsection{Commonly Used Algorithms}
\begin{frame}{Quicksort}
	\begin{itemize}
		\item Fundamentally easy to implement and analyze.
		\item Fast in the average case.
		\item Performs well with (almost) sorted data.
		\item Generally cache friendly.
		\item Good balance between comparisons and swaps.
	\end{itemize}
\end{frame}

\begin{frame}{Mergesort}
	TODO
\end{frame}

\begin{frame}{Insertion Sort}
	TODO
\end{frame}

\section{Exploring Existing Implementations}
\subsection{GNU libc}
\begin{frame}{GNU libc}
	TODO
\end{frame}

\begin{frame}{Wait, \texttt{qsort} doesn't use quicksort?}
	TODO
\end{frame}


% \subsection{Why India?}
% \begin{frame}{Why India?}
% 	\begin{itemize}
% 		\item Ranks second in terms of population worldwide, accounts for 17\% of
% 		      the world's people.
% 		\item Ranks highly in the Renewable Energy Country Attractiveness Index
% 		      (RECAI).
% 		\item Has one of the fastest growing economies in the world.
% 	\end{itemize}
% 	\centering
% % 	\includegraphics[width=7.5cm,keepaspectratio=true]{assets/population.png}
% \end{frame}

% \subsection{Global Comparison}
% \begin{frame}{Global Comparison}
% 	\begin{itemize}
% 		\item Currently the world's \nth{3} largest renewable energy producer.
% 		\item 40\% of energy capacity installed in 2022 comes from renewable
% 		      sources.
% 	\end{itemize}
% 	\centering
% 	% \includegraphics[width=7.5cm,keepaspectratio=true]{assets/global_energy_chart.png}
% \end{frame}

% \subsection{Current Energy Mix}
% \begin{frame}{Current Energy Mix - Overall}
% 	\centering
% 	% \includegraphics[height=7.5cm, keepaspectratio=true]{assets/energy-consumption-india-overall.png}
% \end{frame}

% \begin{frame}{Current Energy Mix - Renewables}
% 	\centering
% 	% \includegraphics[height=7.5cm, keepaspectratio=true]{assets/energy-consumption-india-renewable.png}
% \end{frame}

% \subsection{RECAI}
% \begin{frame}{Renewable Energy Country Attractiveness Index (RECAI)}
% 	\begin{itemize}
% 		\item RECAI ranks the world's top 40 markets on the attractiveness of
% 		      their renewable energy investment and deployment opportunities\cite{owidenergy}.
% 	\end{itemize}

% 	\begin{table}
% 		\begin{tabular}{|c|c|c|}
% 			\hline
% 			Country & Score & RECAI Rank \\
% 			\hline
% 			USA     & 70.7  & 1          \\
% 			China   & 68.2  & 2          \\
% 			India   & 66.2  & 3          \\
% 			\hline
% 		\end{tabular}
% 		\begin{tabular}{|c|c|c|c|}
% 			\hline
% 			Technology       & India & USA  & China \\
% 			\hline
% 			Solar PV         & 62.7  & 57.6 & 60.3  \\
% 			Solar CSP        & 09.2  & 46.2 & 54.3  \\
% 			Hydroelectricity & 46.4  & 57.6 & 60.3  \\
% 			Biofuels         & 47.4  & 45.3 & 52.8  \\
% 			Onshore Wind     & 54.2  & 58.1 & 55.7  \\
% 			Offshore Wind    & 28.6  & 55.6 & 60.6  \\
% 			Geothermal       & 23.2  & 46.0 & 31.7  \\
% 			\hline
% 		\end{tabular}
% 	\end{table}
% \end{frame}

% \section{}
% \subsection{}
% \begin{frame}{Outline}
% 	\tableofcontents
% \end{frame}

% \section{Solar Energy}

% \subsection{Current Development}
% \begin{frame}{Current Development}
% 	\begin{itemize}
% 		\item Solar power generation in India ranks \nth{4} globally in 2021.
% 		\item Rooftop solar accounts for 2.1 GW in 2018.
% 		\item Off-grid solar power for local energy needs is currently under
% 		      development.
% 	\end{itemize}
% 	\centering
% 	% \includegraphics[width=7.5cm,keepaspectratio=true]{assets/rooftop-solar.jpg}
% \end{frame}

% \begin{frame}{}
% 	\begin{itemize}
% 		\item The Indian Government had an initial target of 20 GW of solar
% 		      capacity for 2022, but that was achieved four years ahead of
% 		      schedule.
% 		\item A new goal of 100 GW has been set for 2022, targeting an investment
% 		      of ~\$100 billion.
% 	\end{itemize}
% 	\centering
% 	% \includegraphics[width=7.5cm,keepaspectratio=true]{assets/installed-solar-PV-capacity.png}
% \end{frame}

% \subsection{Solar Potential}
% \begin{frame}{Solar Potential}
% 	\begin{itemize}
% 		\item Approximately 300 clear and sunny days in a year.
% 		\item Daily average solar-power-plant generation capacity in India is 0.30
% 		      kWh per $\textnormal{m}^{2}$.
% 	\end{itemize}
% 	\centering
% 	% \includegraphics[trim=0 10 0 50, clip]{assets/India-PVOut.png}
% \end{frame}

% \subsection{Advantages}
% \begin{frame}{Advantages}
% 	\begin{itemize}
% 		\item Solar products have increasingly helped meet rural demands.
% 		      \begin{itemize}
% 			      \item Substantially reduces dependence on kerosene.
% 		      \end{itemize}
% 		\item Development of inexpensive solar technology could provide a network
% 		      of decentralized local-grid clusters.
% 		      \begin{itemize}
% 			      \item Bypasses expensive, long-distance, power delivery systems.
% 		      \end{itemize}
% 		\item Expansive networks can be built all over India, unlike hydroelectric
% 					power and wind farms.
% 	\end{itemize}
% \end{frame}

% \begin{frame}{Agricultural Support}
% 	\begin{itemize}
% 		\item Photovoltaic water pumping systems
% 		      \begin{itemize}
% 			      \item During hot sunny daytime, increased water demand can be
% 			            met using solar energy.
% 		      \end{itemize}
% 	\end{itemize}
% 	\centering
% 	% \includegraphics[width=7.5cm,keepaspectratio=true]{assets/pv-water-pumping.png}
% \end{frame}

% % TODO
% \subsection{Challenges}
% \begin{frame}{Challenges}
% 	\begin{itemize}
% 		\item Land price is costly. Dedication of land for solar arrays must
% 		      compete with other needs.
% 		\item Production and transport of photovoltaic cells can incur a large
% 		      upfront cost.
% 		      \begin{itemize}
% 			      \item Lack of domestic manufacturing.
% 			      \item Increases dependency on China for PV components.
% 		      \end{itemize}
% 		\item Distribution Losses.
% 	\end{itemize}
% \end{frame}

% \section{Conclusion}
% \begin{frame}{Conclusion}
% 	\begin{itemize}
% 		\item India is a prominent player in the modern energy world.
% 		\item Continued investments in renewable energy sources have been largely
% 		      beneficial for India.
% 		\item Solar energy generation is already gaining attention.
% 		\item Most challenges with solar energy are also faced by other renewable
% 		      sources.
% 	\end{itemize}
% \end{frame}

% \subsection*{Thank You}
% \begin{frame}{}
% 	\centering
% 	\Huge Questions?
% \end{frame}

% \section*{}
% \subsection*{}
% \begin{frame}[allowframebreaks]{References}
% 	\printbibliography
% \end{frame}

\end{document}
